% Summary of using velocity dispersions as an age proxy -- lit review
\subsection{Using kinematics as an age proxy}

The star forming molecular gas clouds observed in the Milky Way have a low
out-of-plane, or vertical, velocity \citep[\eg][]{stark1989, stark2005,
aumer2009, martig2014, aumer2016}.
In contrast, the vertical velocities of older stars are observed to be larger
in magnitude on average \citep{stromberg1946, wielen1977, nordstrom2004,
holmberg2007, holmberg2009, aumer2009, casagrande2011, ting2019, yu2018}.
There are two possible explanations for this observed increase in velocity
dispersion with age: either stars are born kinematically `cool' and their
orbits are heated over time via interactions with giant molecular clouds
\citep[see][for a review of secular evolution in the MW]{sellwood2014}, or
stars formed kinematically `hotter' in the past \citep[\eg][]{bird2013}.
Either way, the vertical velocity dispersions of thin disk stars are observed
to increase with stellar age.
This behavior is codified by Age-Velocity dispersion Relations (AVRs), which
typically express the relationship between age and velocity dispersion as a
power law: $\sigma_v \propto t^\beta$, with free parameter, $\beta$
\citep[\eg][]{holmberg2009, yu2018}.
These expressions can be used to infer the ages of groups of stars from their
velocity dispersions, as we do in this paper (see section \ref{sec:method}).

Kinematic ages have been used to explore the evolution of cool dwarfs for over
a decade.
\citet{west2004, west2006} found that the fraction of magnetically active M
dwarfs decreases over time, by using the vertical distances of stars from the
Galactic mid-plane as an age proxy, and \citet{west2008} used kinematic ages
to calculate the expected activity lifetime for M dwarfs of different spectral
types.
\citet{faherty2009} used tangential velocities to infer the ages of M, L and T
dwarfs, and showed that dwarfs with lower surface gravities tended to be
kinematically younger, and \citet{kiman2019} used velocity dispersion as an
age proxy to explore the evolution of H$\alpha$ equivalent width (a magnetic
activity indicator), in M dwarfs.

% After the development of a radiative core, little angular momentum is
% transported between the core and convective envelope and, if wind-braking
% slows the surface substantially before the two zones recouple, the core may
% rotate much more quickly than the envelope.
% The Sun rotates almost as a solid body \citep[\eg][]{thompson1996}, so it is
% expected that the cores and envelopes of Solar-like stars eventually
% re-couple, with angular momentum efficiently transported between them.
% The timescales for recoupling have been studied extensively and explored in
% theoretical models for decades \citep[\eg][]{endal1981, macgregor1991,
% denissenkov2010, gallet2013, lanzafame2015}.

AVRs are usually calibrated in Galactocentric velocity coordinates (\vx, \vy,
\vz\ or $UVW$), and these velocities can only be calculated with full 6D
positional and velocity information, however most \kepler\ rotators do not
have RV measurements\footnote{Although RVs for most will be released in \gaia\
DR3}.
In Angus \etal\ (2020) we used velocity in the direction of Galactic latitude
(\vb) as a stand-in for \vz\ because, in the {\it Galactic} coordinate system,
velocities can be calculated from 3D positions and {\it 2D} proper motions.
The \kepler\ field lies at low Galactic latitude, so \vb\ is a close
approximation to \vz.
Though \vb\ velocity dispersion does not equal \vz\ velocity dispersion, it
still increases monotonically over time and provides accurate age rankings for
\kepler\ stars.
Unfortunately however, given that AVRs are calibrated in {\it Galactocentric}
coordinates (\vx, \vy, \vz), we could not directly translate \vb\ velocity
dispersions to ages.

In this paper, our aim was to use kinematic ages to calibrate a new
gyrochronology relation, for which four main steps were required.
Firstly, we inferred {\it vertical} velocity, \vz, for each star without an RV
measurement by marginalizing over missing RVs using a hierarchical Bayesian
model (see section \ref{sec:velocity_inference}).
Secondly, we calculated velocity dispersion for every star using a moving, or
rolling dispersion method (see section \ref{sec:velocity_dispersion}).
Thirdly, these velocity dispersions were converted into ages using an AVR
\citep[][section \ref{sec:avr}]{yu2018}.
Finally, we used a Gaussian process model to capture the complexities of
stellar rotational evolution and calibrated a new gyrochronology relation
using our kinematic ages, plus benchmark cluster and asteroseismic stars in
section \ref{sec:gp_model}.
