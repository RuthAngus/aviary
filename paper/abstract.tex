% Context
% Gyrochronology, the method of inferring star's age from its rotation period,
% has great potential as a dating method for low-mass (GKM) stars.
% However, the small number of available calibration stars with precisely
% measured rotation periods and ages means that the gyrochronology relations
% still suffer from poor calibration.
% % Aims
% We used kinematics to infer the ages of field GKM dwarfs, observed by Kepler,
% with the goal of calibrating gyrochronology over a broad range of stellar
% masses and ages.
% % Method
% We first calibrated a new age-velocity dispersion relation using asteroseismic
% and cluster stars.
% This relation was then used to infer ages for around 6000 Kepler field stars
% with measured rotation periods.
% These ages were compared with ages from the literature, calculated using
% stellar evolution models.
% We find good agreement?
% Finally, using these kinematic ages, we fit a Gaussian process to the rotation
% periods, effective temperatures and ages of these field stars, producing a new
% empirical gyrochronology model.
% This model is available for use in the {\tt stardate} \python\ package.

Gyrochronology, the method of inferring the age of a star from its rotation
period, could provide ages for billions of stars over the coming decade of
time-domain astronomy.
However, the gyrochronology relations remain poorly calibrated due to a lack
of precise ages for old, cool main-sequence stars.
Now however, with proper motion measurements from Gaia, Galactic kinematics
can be used as an age proxy, and the magnetic and rotational evolution of
stars can be examined in detail.
We demonstrate that kinematic ages, inferred from the velocity dispersions of
groups of stars, beautifully illustrate the time and mass-dependence of the
gyrochronology relations.
We use kinematic ages of field stars, plus benchmark clusters and
asteroseismic stars, to calibrate a new empirical, Gaussian process
gyrochronology relation, that fully captures the complex rotational evolution
of cool dwarfs over a range of masses and ages.
We use cross validation to demonstrate that this relation accurately predicts
ages for FGKM dwarfs.
