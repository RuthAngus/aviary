\section{Method}

\subsection{Data}
\label{sec:data}

This study focuses on stellar rotation in the original \kepler\ field, partly
because \kepler\ provides the largest sample of published rotation periods,
and partly because its low Galactic latitude allows us to marginalize over
missing RV measurements to approximate \vz.
We combined three rotation period catalogs constructed from original \kepler\
data: \citet{mcquillan2014}, \citet{santos2019} and \citet{garcia2014}.

\subsection{Inferring 3D velocities (marginalizing over missing RV
measurements)}

It has been demonstrated that the dispersion in vertical velocity, \vz\, for a
group of stars increases with the age of that group (citations).
However, velocities in Galactocentric coordinates, \vx, \vy\ and \vz, can only
be calculated with full 6-D position and velocity information, \ie\ proper
motions, position and radial velocity.
In \citet{angus2020} we introduced the idea that kinematic ages could be used
to calibrate gyrochronology and showed, in the appendix of that paper, that
velocity, \vb\ in the Galactic frame, which can be calculated without an RV
measurement, can be used as an approximation to \vz\ for \kepler\ stars.
This is because the \kepler\ field of view lies at relatively low Galactic
latitudes, ($\sim 5-20$\degrees), so the $z$-direction is similar to the
$b$-direction for \kepler\ stars.
However, \vb\ is only a close approximation to \vz\ at extremely {\it low}
latitudes, and even in the \kepler\ field, kinematic ages calculated with \vb\
instead of \vz\ are systematically larger because of extra noise introduced by
the imperfect translation between \vb\ and \vz\.
In order to calculate accurate vertical velocities and therefore ages, the
appropriate approach is to {\it infer} \vz\  by marginalizing over missing RV
measurements.

Three-dimensional velocities in galactocentric coordinates: \vx, \vy, and \vz\
can only be directly computed via a transformation from 3D velocities in
another coordinate system, like the equatorial coordinates provided by \gaia:
\mura, \mudec, and RV.
For stars with no measured RV in \gaia\ DR2, \vx, vy, and \vz\ can still be
inferred from positions and proper motions alone, by marginalizing over
missing RV measurements.
For each star in our sample, we inferred \vx, \vy, and \vz\ from the 3D
positions and proper motions provided in the \gaia\ DR2 catalog
\citep{brown2011}.
We also simultaneously inferred distance, instead of using 1/\parallax, to
model velocities \citep{bailer-jones2016}.

Using Bayes rule, the posterior probability of the parameters given the data
can be written:
\begin{equation}
p(v_{\bf xyz}, D | \mu_{\alpha}, \mu_{\delta}, \alpha, \delta, \pi) =
    p(\mu_{\alpha}, \mu_{\delta}, \alpha, \delta, \pi | v_{\bf xyz}, D)
    p(v_{\bf xyz}) p(D),
\end{equation}
where D is distance, $\alpha$ is Right Ascension (RA), $\delta$ is declination
(dec), $\pi$ is parallax, $\mu_\alpha$ is proper motion in RA, and
$\mu_\delta$ is proper motion in dec.
The prior over $\log$(distance) and velocities was a multivariate Gaussian
with mean and covariance determined from the distance and velocity
distributions of \kepler\ targets with RV measurements.

The posterior PDF was explored using {\tt emcee} \citep{forman-mackey2013}, an
affine-invariant, ensemble MCMC sampler.

Initialization.

We found that 10,000 samples with 16 walkers was sufficient to calculate a
converged autocorrelation time, and produce 150-500 independent samples per
parameter.

Over 3000 stars in the \mct\ sample do have RV measurements and provide an
opportunity to test this method of inferring velocities.
Figure \ref{fig:v_comparison} shows the velocities of these 3000 stars,
calculated using RV measurements, compared with their inferred velocities.

\begin{figure}[ht!]
\caption{Vertical velocities calculated with full 6D information vs vertical
    velocities inferred without RV, for all 3000 \mct\ stars with \gaia\ RV
    measurements.}
  \centering
    \includegraphics[width=1\textwidth]{v_comparison}
\label{fig:v_comparison}
\end{figure}

\subsection{Calculating velocity dispersions}
A kinematic age can be calculated from the velocity dispersion (\eg\ the
standard deviation or Median Absolute Deviation, MAD, of velocities) of a
group of stars.
Many empirical Age-Velocity dispersion Relations (AVRs) have been calibrated
using stars with measured proper motions, radial velocities, and ages
\citep[\eg][]{holberg2009, yu2019}.
The major assumption underlying kinematic ages is that the stars used to
calculate a velocity dispersion are {\it all the same age}.
So, in order to calculate kinematic ages from velocity dispersions for
\kepler\ stars, it is necessary to group them by age.

\subsection{Converting velocity dispersion to age with an AVR}

\subsection{Comparing kinematic ages with asteroseismic and cluster ages}

\subsection{A Gaussian process gyrochronology relation}
