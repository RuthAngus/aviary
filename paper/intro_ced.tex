% % Core-envelope coupling
% A period of core-envelope decoupling is necessary to explain the observed
% rotation periods of stars in extremely young open clusters (1-10 Gyr)
% \citep[\eg][]{irwin2007, bouvier2008, denissenkov2010, spada2011, reiners2012,
% gallet2013}.
% After the development of a radiative core, little angular momentum is
% transported between the core and convective envelope and, if wind-braking
% slows the surface substantially before the two zones recouple, the core may
% rotate much more quickly than the envelope.
% The Sun rotates almost as a solid body \citep[\eg][]{thompson1996}, so it is
% expected that the cores and envelopes of Solar-like stars eventually
% re-couple, with angular momentum efficiently transported between them.
% The timescales for recoupling have been studied extensively and explored in
% theoretical models for decades \citep[\eg][]{endal1981, macgregor1991,
% denissenkov2010, gallet2013, lanzafame2015}.
% The physical mechanism responsible for internal angular transport is still
% unknown, however magnetic field-induced coupling and gravity waves are two
% processes often used to explain the phenomenon \citep[see,
% \eg][]{charbonneau1993, ruediger1996, spruit2002, talon2003, spada2010,
% brun2011, oglethorpe2013}.
% Based on observations of lithium depletion and rotation period evolution in
% open clusters, including new rotation period measurements for the $\sim$1 Gyr
% NGC6811 cluster \citep{curtis2019}, the coupling timescale has been determined
% to be strongly mass-dependent \citep{lanzafame2015, somers2016, spada2019}.
% Semi-empirical models with mass-dependent angular momentum transport are able
% to reproduce the rotation periods in open clusters and the field
% \citep{spada2019, angus2020}.

\subsection{Core-envelope decoupling}

% Describe previous paper
In Angus \etal\ (2020) we demonstrated that Galactic kinematics can be used to
explore the evolution of stellar rotation.
We showed that velocity dispersion, an established age proxy in the Galactic
thin disk, increases smoothly as a function of rotation period, indicating
that rotation period increases with age as expected.
Using velocity dispersion as an age proxy, we also showed that old K dwarfs
spin down more slowly than G dwarfs: their rotational evolution appears to
`stall' after around 1 Gyr, in a manner that reflects the behavior of K dwarfs
observed in open clusters \citep{curtis2019}.
At young ages ($\sim$ 0.5 -- 1 Gyr), K dwarfs spin more slowly than G dwarfs of
the same age, because their deeper convection zones generate stronger magnetic
fields, which leads to more efficient magnetic braking.
However, at old ages ($\gtrsim$ 1 Gyr) K dwarfs rotate at the same rate or
more rapidly than contemporary G dwarfs.
The leading explanation for this phenomenon is that angular momentum is
transferred from the core to the surface over longer timescales for lower-mass
stars \citep{spada2019}, \ie\ they experience a more extended phase of
`core-envelope decoupling'.

% The angular momentum of the radiative cores and convective envelopes of stars
% are thought to evolve separately at young ages \citep[\eg][]{mcdonald1995,
% gallet2013}.
A period of core-envelope decoupling is necessary to explain the observed
rotation periods of stars in extremely young open clusters (1-10 Gyr)
\citep[\eg][]{irwin2007, bouvier2008, denissenkov2010, spada2011, reiners2012,
gallet2013}.
During this phase there is little transfer of angular momentum between
radiative core and convective envelope and, as wind-braking removes angular
momentum from the envelope, it decelerates while the core continues to spin
rapidly.
Over time however, angular momentum is transported across the interface
between the two zones, and momentum from the rapidly spinning interior
surfaces, inhibiting the deceleration of the outer envelope.
% With a mass-dependent timescale for core-envelope coupling semi-empirical
% models are able to reproduce the
Currently, the rotation periods of field and cluster stars can only be
reproduced by semi-empirical models with a mass-dependent timescale for
core-envelope coupling \citep[][Angus \etal, 2020]{spada2019, curtis2019}.

