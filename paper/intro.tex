\section{Introduction}

% Summary of gyrochronology
Stars with significant convective envelopes ($\lesssim$ 1.3 M$_\odot$) have
strong magnetic fields and slowly lose angular momentum via magnetic braking
\citep[\eg][]{schatzman1962, weber1967, skumanich1972, kawaler1988,
pinsonneault1989}.
Although stars are born with random rotation periods, from 1 to 10 days,
observations of young open clusters reveal that their rotation periods
converge onto a unique sequence by $\sim$500-700 million years
\citep[\eg][]{irwin2009, gallet2013}.
After this time, the rotation period of a star is thought to be determined, to
first order, by its color and age alone.
This is the principle behind gyrochronology, the method of inferring a
star’s age from its rotation period \citep[\eg][]{barnes2003, barnes2007,
barnes2010, meibom2011, meibom2015}.

The rotational evolution of stars is clearly a complicated process and, to
fully calibrate the gyrochronology relations we need a large sample of
reliable ages for stars spanning a range of ages and masses.
In this paper, we use the velocity dispersions of field stars to qualitatively
explore the rotational evolution of GKM dwarfs, and show that kinematics could
provide a gyrochronology calibration sample.

% Summary of using velocity dispersions as an age proxy -- lit review
\subsection{Using kinematics as an age proxy}

% Describe previous paper
In \citet{angus2020} we demonstrated that kinematics can be used to explore
the evolution of stellar rotation.
In this paper we take the next step and use kinematics to calibrate a new
gyrochronology relation.
This paper is laid out as follows.
In section \ref{sec:kinematics} we describe the calibration of a new AVR, and
how we used it to calculate ages for over 6000 stars with measured rotation
periods.
In this section we also compare these kinematc ages with literature age
measurements.
In section \ref{sec:gyro} we describe how we use these kinematic ages to
calibrate a new empirical gyrochronology relation.
