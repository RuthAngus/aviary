\section{Introduction}

% Motivation
Low mass dwarfs are the most common stars in the Milky Way, and their ages
could reveal the evolution of Galactic stellar populations and planetary
systems.
However, the ages of GKM stars are difficult to measure because their
luminosities and temperatures evolve slowly on the main sequence.
Fortunately, rotation-dating, or `gyrochronology’ provides a promising means
to measure precise ages for these cool dwarfs.
The rotation periods of these stars evolve relatively rapidly, and a fully
calibrated gyrochronology model that captures the time and mass-dependence of
stellar spin down could provide ages that are precise to within 20\% for
millions of Milky Way stars in the time-domain era \citep{epstein2014,
najita2016, angus2019}.

% What do we still not understand?
% Gyro is still poorly calibrated
With the thousands of new photometric rotation period measurements provided by
specialized ground and space-based missions \citep[particularly \kepler/\ktwo\
and \tess][]{borucki2010, howell2014, ricker2015}, we are making progress
towards the ultimate goal for rotation-dating: a fully calibrated
gyrochronology relation, applicable to GKM main-sequence stars of all ages.
A lack of low-mass and old calibration stars has previously limited the mass
and age coverage of gyrochronology relations, which can only be calibrated
using stars with precise age and rotation period measurements.
Historically, the calibration sample has been limited to open clusters and
asteroseismic stars.
 % because cluster members and Solar-like oscillators can be
% precisely dated via main-sequence turn off and asteroseismology.
Open clusters provide good mass coverage for young stars: rotation periods
have been measured for F through mid M dwarfs for stars in clusters with
precisely measured ages up to around 700 Myr.
Asteroseismic stars provide reasonable age coverage for hot stars: seismic
masses and photometric surface rotation periods have been measured for F, G
and early K dwarfs for stars as old as 10 Gyr.
However, neither asteroseismology nor cluster analysis can provide rotation
periods and ages for old, late K and M dwarfs.
In addition, cluster and asteroseismic stars generally provide sparse coverage
of the rotation period-effective temperature plane, and cannot reveal the
detailed evolution of stellar rotation rates.
As a result, most empirical gyrochronology relations are only reliable for G
dwarfs up to Solar age, K dwarfs up to 2-3 Gyr, and early M dwarfs up to $<$ 1
Gyr.
% Old open clusters are generally too distant for photometric rotation period
% measurements of faint M dwarfs, and asteroseismic analyses suffer from the
% large magnetic activity signals and low-amplitudes of oscillations for
% low-mass stars.
For this reason, the rotational evolution of cool dwarfs is not well
understood.
However, as we showed in \citet{angus2020}, {\it kinematic} ages can be used
to turn field stars, observed by \kepler, into a calibration sample that
provides good mass and age coverage.
Although the \kepler\ sample does not include late M dwarfs, it can still be
used to extend gyrochronology relations to much older ages for late K and
early M dwarfs.
By adding the ages and rotation periods of thousands of field stars to the
open cluster and asteroseismic calibration sample, we can calibrate a
gyrochronology relation that is applicable to FGK and early M dwarfs between
the ages of $\sim$500 Myr and 8 Gyr.

% % Gyro 101
% Stars with significant convective envelopes ($\lesssim$ 1.3 M$_\odot$) have
% strong magnetic fields which are thought to be generated by a Solar-type
% $\alpha-\Omega$ dynamo.
% These stars lose angular momentum over long timescales via magnetic braking
% \citep[\eg][]{schatzman1962, weber1967, kraft1967, skumanich1972, kawaler1988,
% pinsonneault1989}.
% Stars arrive on the main sequence with a random distribution of rotation
% periods, ranging from 1 oto 10 days \citep{rebull2019}, however the rotation
% periods of stars in open clusters have converged onto a unique age-rotation
% period-mass sequence by $\sim$500-700 million years \citep[\eg][]{irwin2009,
% gallet2013}.
% After this time, the rotation period of a star is thought to be determined, to
% first order, by its photometric color/effective temperature/mass and age
% \citep[\eg]{barnes2003, barnes2007, barnes2010, meibom2011, meibom2015}.

% The rate at which a star loses angular momentum is thought to chiefly depend
% on its magnetic field strength.
% Lower-mass stars with deeper convection zones have stronger magnetic fields,
% thus larger Alfvén radii, and therefore experience greater angular momentum
% loss rates \citep[\eg][]{schatzman1962, kraft1967, parker1970, kawaler1988,
% charbonneau2010}.
% This effect dominates the evolutionary sequence of young stars and, for
% example in the $\sim$ 700 Myr Praesepe cluster, rotation rate increases with
% increasing mass (rotation {\it period decreases} with increasing mass).
% However, at ages greater than around 1 Gyr \citep{spada2019, curtis2019,
% angus2020}, internal angular momentum transport becomes important and starts
% to influence the evolution of surface rotation periods.

% % Core-envelope coupling
% A period of core-envelope decoupling is necessary to explain the observed
% rotation periods of stars in extremely young open clusters (1-10 Gyr)
% \citep[\eg][]{irwin2007, bouvier2008, denissenkov2010, spada2011, reiners2012,
% gallet2013}.
% After the development of a radiative core, little angular momentum is
% transported between the core and convective envelope and, if wind-braking
% slows the surface substantially before the two zones recouple, the core may
% rotate much more quickly than the envelope.
% The Sun rotates almost as a solid body \citep[\eg][]{thompson1996}, so it is
% expected that the cores and envelopes of Solar-like stars eventually
% re-couple, with angular momentum efficiently transported between them.
% The timescales for recoupling have been studied extensively and explored in
% theoretical models for decades \citep[\eg][]{endal1981, macgregor1991,
% denissenkov2010, gallet2013, lanzafame2015}.
% The physical mechanism responsible for internal angular transport is still
% unknown, however magnetic field-induced coupling and gravity waves are two
% processes often used to explain the phenomenon \citep[see,
% \eg][]{charbonneau1993, ruediger1996, spruit2002, talon2003, spada2010,
% brun2011, oglethorpe2013}.
% Based on observations of lithium depletion and rotation period evolution in
% open clusters, including new rotation period measurements for the $\sim$1 Gyr
% NGC6811 cluster \citep{curtis2019}, the coupling timescale has been determined
% to be strongly mass-dependent \citep{lanzafame2015, somers2016, spada2019}.
% Semi-empirical models with mass-dependent angular momentum transport are able
% to reproduce the rotation periods in open clusters and the field
% \citep{spada2019, angus2020}.

% Summary of using velocity dispersions as an age proxy -- lit review
\subsection{Using kinematics as an age proxy}

% Describe previous paper
In Angus \etal\ (2020) we demonstrated that Galactic kinematics can be used to
explore the evolution of stellar rotation.
In this paper we take the next step and use kinematic ages to calibrate a new
gyrochronology relation.

In general, the velocity dispersions of stars in the Galactic thick disk are
observed to increase with age \citep[\eg][]{casagrande2011, aumer2009}

This paper is laid out as follows.
In section \ref{sec:kinematics} we describe the calibration of a new AVR, and
how we used it to calculate ages for over 6000 stars with measured rotation
periods.
In this section we also compare these kinematc ages with literature age
measurements.
In section \ref{sec:gyro} we describe how we use these kinematic ages to
calibrate a new empirical gyrochronology relation.
