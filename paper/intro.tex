\section{Introduction}

% Motivation
Main-sequence dwarfs are the most common stars in the Milky Way, and their
ages could reveal the evolution of Galactic stellar populations and planetary
systems.
However, the ages of low-mass stars, particularly K and M dwarfs are difficult
to measure because their luminosities and temperatures evolve slowly on the
main sequence \citep[see][for a review of stellar ages]{soderblom2010}.
Fortunately, rotation-dating, or `gyrochronology’ provides a promising means
to measure precise ages for these cool dwarfs
\citep[\eg][]{schatzman1962, weber1967, kraft1967, skumanich1972, kawaler1988,
pinsonneault1989, barnes2003, barnes2007, mamajek2008, barnes2010, meibom2011,
epstein2014, meibom2015, vansaders2016, vansaders2018, claytor2020}.
% Gyro 101
Stars with significant convective envelopes ($\lesssim$ 1.3 M$_\odot$) have
strong magnetic fields, generated by a Solar-type $\alpha-\Omega$ dynamo.
These stars lose angular momentum over long timescales via magnetic braking
\citep[\eg][]{schatzman1962, weber1967, kraft1967, skumanich1972, kawaler1988,
pinsonneault1989}.
Stars arrive on the main sequence with a random distribution of rotation
periods, ranging from around 1 to 10 days \citep{rebull2019}, however
observations of open clusters show that stellar rotation periods
converge onto a unique sequence by around 500-700 million years
\citep[\eg][]{irwin2009, gallet2013}.
This sequence is often called the `slow rotator sequence'.
After stars reach the slow rotator sequence, the rotation period of a star can
be determined, to first order, by its photometric color and age
\citep[\eg]{barnes2003, barnes2007, barnes2010, meibom2011, meibom2015}.

A well-calibrated gyrochronology model that captures the time and
mass-dependence of stellar angular momentum evolution could provide ages that
are precise to within 20\% for millions of Milky Way stars in the time-domain
era \citep{epstein2014, najita2016, angus2019, claytor2020}.
However, gyrochronology models are not yet fully calibrated, especially for K
and M dwarfs, as they are limited by a lack of low-mass stars with precisely
measured ages and rotation periods.
For example, the oldest K dwarfs in an open cluster with measured rotation
periods are those in Ruprecht 147 (2.7 Gyr) \citep{curtis2020, age_citation}.
The oldest cluster M dwarfs with measured rotation periods are those in
Praesepe (600-700 Myr) \citep{douglas2017, rebull2017, age_citation}.
Gyrochronology relations can therefore be considered untested for K dwarfs
older than 2.7 Gyr and M dwarfs older than 600-700 Myr.

Once stars have converged to the slow rotator sequence, their rotation periods
generally follow a power law relation in time with an index of around 0.5
\citep{skumanich1972}.
However, new Kepler/K2 rotation periods of stars in open clusters and
asteroseismic stars, show that the exact value of this power law index varies
as a function of mass and age \citep{angus2015, vansaders2016, curtis2019,
spada2019}.
Two major phenomena in particular, `weakened magnetic braking' and `stalled
braking' are both responsible for slowing the rate of magnetic
braking and create departures from the \citet{skumanich1972} relation.
These two phenomena affect stars of different masses and ages and are thought
to be generated by different physical processes.
`Weakened magnetic braking' was first observed in asteroseismic stars with
rotation rates that appeared too rapid for their age
\citep{angus2015}.
It affects the rotational evolution of G dwarfs older
than around Solar age (4-5 Gyrs) and is thought to be caused by a shutting down
of the magnetic dynamo when stars reach a Rossby number of around 2
\citep{vansaders2016, vansaders2018}.
\footnote{Rossby number is the ratio of rotation period to convective overturn
time.}
`Stalled braking' is a different behavior that has been observed in
the K dwarfs of at least three open clusters: NGC 752, NGC 6811, and Ruprecht
147 \citep{agueros, curtis2019, curtis2020}.
The K dwarfs in these clusters again appear to rotate too rapidly for their
age, however in this case the underlying physical cause is thought to be the
redistribution of angular momentum in the stellar interior \citep{spada2019}.

With the discovery of weakened and stalled braking, it is clear that stars of
different masses and ages spin-down at different rates.
Because of the non-linear and time-variable relation between rotation period
and age, the gyrochronology relations should not, if possible, be extrapolated
to old ages or low masses where little calibration data exist.
Instead, they should be actively calibrated for stars of all masses and ages.
Our goal in this paper is to provide a new fully-empirical gyrochronology
relation that captures the stalled braking behavior of old K and early M
dwarfs.
However, our model will not capture the behavior of stars undergoing weakened
braking because our calibration is not yet suitable for weakened braking
stars.
We hope to extend our model to these stars in future

% The rate at which a star loses angular momentum is thought to chiefly depend
% on its magnetic field strength.
% Lower-mass stars with deeper convection zones have stronger magnetic fields,
% thus larger Alfvén radii, and therefore experience greater angular momentum
% loss rates \citep[\eg][]{schatzman1962, kraft1967, parker1970, kawaler1988,
% charbonneau2010}.
% This effect dominates the evolutionary sequence of young stars and, for
% example in the $\sim$ 700 Myr Praesepe cluster, rotation rate increases with
% increasing mass (rotation {\it period decreases} with increasing mass).
% However, at ages greater than around 1 Gyr \citep{spada2019, curtis2019,
% angus2020}, internal angular momentum transport becomes important and starts
% to influence the evolution of surface rotation periods.

% The rotation periods of these calibration stars can often be measured with
% precise time-series photometry, and the Kepler spacecraft has played a leading
% role in providing light curves for stellar rotation studies
% \citep[\eg][]{borucki2010, meibom2011, mcquillan2013, mcquillan2014,
% howell2014, reinhold2014, garcia2014, aigrain2015, meibom2015, douglas2016,
% douglas2017, rebull2016, rebull2017, santos2019, reinhold2019, rebull2020,
% gordon2020, curtis2020, breton2021}.
% Magnetically active regions on the surfaces of stars create dark and bright
% surface features which produce periodic variability in the overall brightness
% of a rotating star.
% This variability is, at most, on the order of 1\% of the star's overall flux.
% The rotation periods of stars can often be measured from their light curves
% using signal-processing techniques.
% TESS, the Transiting Exoplanet Survey Satellite, is continuing the legacy of
% Kepler and will provide rotation periods for many more stars
% \citep{ricker2014}.

% For the purposes of calibrating gyrochronology, open clusters provide good
% mass coverage for young stars: rotation periods have been measured for F to
% mid M dwarfs up to ages of around 700 Myr.
% In contrast, asteroseismic stars provide reasonable age coverage for hot
% stars: ages and photometric surface rotation periods have been measured for F,
% G and early K dwarfs up to ages of 10 Gyr.
% However, neither asteroseismology nor cluster analysis can provide rotation
% periods and ages for old, late K and M dwarfs.
% In addition, cluster and asteroseismic stars generally provide sparse coverage
% of the rotation period-effective temperature plane, and cannot reveal the
% detailed evolution of stellar rotation rates.
% As a result, most empirical gyrochronology relations are only reliable for G
% dwarfs up to Solar age, K dwarfs up to 2-3 Gyr, and early M dwarfs up to $<$ 1
% Gyr.

Historically, gyrochronology calibration samples have mainly been comprised of
stars in open clusters and stars with detectable acoustic pulsations, both of
which can be precisely dated with stellar evolution models.
However, to extend gyrochronology relations to older K and M dwarfs, it may be
necessary to use alternative dating methods.
Old open clusters are generally too distant for photometric rotation period
measurements of faint M dwarfs, and asteroseismic analyses are impacted by the
large magnetic activity signals and low-amplitudes of oscillations for
low-mass stars.
There is great promise, however, in using binaries as an alternative method
for calibrating gyrochronology.
For example, low-mass K and M dwarfs with a precisely dateable co-moving
companion (\eg\ a white dwarf, a subgiant, or a star with detectable acoustic
oscillations) could be good calibrators.
In addition, Galactic kinematics is a dating method with good potential for
gyrochronology calibration.
In \citet{angus2020} we explored the kinematic properties of Kepler field
stars and found that the velocity dispersion of stars increase with stellar
rotation period, as expected (because both quantities increase with age).
In \citet{lu2021} we used the velocity dispersions of Kepler field stars to
estimate their ages using an Age-Velocity dispersion Relation (AVR).
In this paper, we use those kinematic ages for around 30,000 Kepler field
stars to extend the gyrochronology relations into the old K and early M dwarf
regime.

