\section{The Data}
\label{sec:data}

This study focuses on stellar rotation in the original \kepler\ field, partly
because \kepler\ provides the largest samples of homogeneously measured
rotation periods, and partly because its low Galactic latitude allows us to
marginalize over missing RV measurements and precisely infer vertical
velocity, \vz.
We combined two large rotation period catalogs constructed from original
\kepler\ data: \mct\ and \sant.
These two studies used different techniques to measure rotation periods from
\kepler\ light curves: autocorrelation functions and wavelets respectively.
The \citet{santos2019} study was specifically focused on cooler stars: K and M
dwarfs, and includes a larger number of rotation periods for these stars.
The combined catalogs provide a total of over 38,000 rotation periods.

We used the publicly available \kepler-\gaia\ DR2 crossmatched
catalog\footnote{Available at gaia-kepler.fun} to combine the \mct\ and \sant\
rotation catalogs with the \gaia\ DR2 catalog of parallaxes, proper motions
and apparent magnitudes.
Reddening and extinction from dust was calculated for each star using the
Bayestar dust map implemented in the {\tt dustmaps} {\it Python} package
\citep{green2018}, and {\tt astropy} \citep{astropy2013, astropy2018}.
We used \gaia\ DR2 photometric color, $G_{\rm BP} - G_{\rm RP}$, to estimate
effective temperatures for the stars in our sample, using the calibrated
relation in \citet{curtis2020}.

Unlike isolated main-sequence stars, the rotation periods of binary stars and
subgiants cannot always be determined by their mass and age (or at least they
do not always follow the {\it same} gyrochronology relationship as isolated
dwarfs).
Photometric binaries and subgiants were therefore removed from the sample by
applying cuts to the color-magnitude diagram (CMD), shown in figure
\ref{fig:CMD}.
A 6th-order polynomial was fit to the main sequence and raised by 0.27 dex to
approximate the division between single stars and photometric binaries (shown
as the curved dashed line in figure \ref{fig:CMD}).
All stars above this line were removed from the sample.
Potential subgiants were also removed by eliminating stars brighter than 4th
absolute magnitude in \gaia\ G-band.
This cut also removed a number of main sequence F stars from our sample,
however these hot stars are not the focus of our gyrochronology study since
their small convective zones inhibit the generation of a strong magnetic
field.
The removal of photometric binaries and evolved/hot stars reduced the total
sample of around 38,000 stars by around 4,000.

3587 stars in our sample had RV measurements available in \gaia\ DR2, with a
median uncertainty of 1.88 \kms.
\gaia\ DR2 included RVs for stars with \gaia\ apparent magnitudes between 4
and 13, and 3550 K $\lesssim$ \teff\ $\lesssim$ 6900 K \citep{brown2018}.
We also crossmatched the \mct\ sample with the 5th LAMOST data release
\citep{cui2012, xiang2019}, adding a further 7466 RV measurements to the
sample, and expanding the total number of stars with measured RVs to 11,053.
The median uncertainty of the LAMOST RV measurements was 4.71 \kms\ and,
given that the \gaia\ RVs were more precise, on average, than the LAMOST
RVs, we adopted the \gaia\ value in cases where both were available.
% \gaia\ DR3 will contain a large number of new RV measurements for stars in our
% sample.
