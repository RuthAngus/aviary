% zip files.zip ms63.tex CMD_cuts_double.pdf main_figure.pdf gap.pdf vb_vs_teff.pdf vb_vz.pdf field_comparison.pdf bib.bib ms.pdf

\documentclass{aastex63}

% Latin
\newcommand{\ie}{{\it i.e.}}
\newcommand{\eg}{{\it e.g.}}
\newcommand{\etal}{{\it et al.}}

% Missions & Spacecraft
\newcommand{\kepler}{{\it Kepler}}
\newcommand{\Kepler}{{\it Kepler}}
\newcommand{\corot}{{\it CoRoT}}
\newcommand{\Ktwo}{{\it K2}}
\newcommand{\ktwo}{\Ktwo}
\newcommand{\TESS}{{\it TESS}}
\newcommand{\tess}{{\it TESS}}
\newcommand{\LSST}{{\it LSST}}
\newcommand{\lsst}{{\it LSST}}
\newcommand{\Wfirst}{{\it WFIRST}}
\newcommand{\wfirst}{{\it WFIRST}}
\newcommand{\SDSS}{{\it SDSS}}
\newcommand{\PLATO}{{\it PLATO}}
\newcommand{\plato}{{\it PLATO}}
\newcommand{\Gaia}{{\it Gaia}}
\newcommand{\gaia}{{\it Gaia}}
\newcommand{\panstarrs}{{\it PanSTARRS}}
\newcommand{\LAMOST}{{\it LAMOST}}

% Units and Quantities
\newcommand{\Teff}{$T_{\mathrm{eff}}$}
\newcommand{\teff}{$T_{\mathrm{eff}}$}
\newcommand{\FeH}{[Fe/H]}
\newcommand{\feh}{[Fe/H]}
\newcommand{\prot}{$P_{\mathrm{rot}}$}
\newcommand{\pmega}{$\bar{\omega}$}
\newcommand{\logg}{log(g)}
\newcommand{\dnu}{$\Delta \nu$}
\newcommand{\numax}{$\nu_{\mathrm{max}}$}
\newcommand{\degrees}{$^\circ$}
\newcommand{\vx}{$v_{\bf x}$}
\newcommand{\vy}{$v_{\bf y}$}
\newcommand{\vz}{$v_{\bf z}$}
\newcommand{\vb}{$v_{\bf b}$}
\newcommand{\kms}{kms$^{-1}$}
\newcommand{\sigmavb}{$\sigma_{v{\bf b}}$}
\newcommand{\sigmavz}{$\sigma_{v{\bf z}}$}

% Paper-specific and other
\newcommand{\logp}{$\log(P_\mathrm{rot})$}
\newcommand{\mura}{$\mu_\alpha$}
\newcommand{\mudec}{$\mu_\delta$}
\newcommand{\parallax}{$\pi$}
\newcommand{\gcolor}{$G_{BP} - G_{RP}$}
\newcommand{\mcp}{\citep{mcquillan2014}}
\newcommand{\mct}{\citet{mcquillan2014}}
\newcommand{\bvector}{${\bf b}$}
\newcommand{\python}{{\it Python}}
\newcommand{\racomment}[1]{{\color{blue}#1}}

% \shorttitle{Calibrating gyrochronology with kinematics}
% \shortauthors{Angus \etal}

\begin{document}

\title{Calibrating gyrochronology using Galactic kinematics}

% \correspondingauthor{Ruth Angus}
% \email{rangus@amnh.org}

\author{Ruth Angus}
\affiliation{Department of Astrophysics, American Museum of Natural History,
200 Central Park West, Manhattan, NY, USA}
\affiliation{Center for Computational Astrophysics, Flatiron Institute,
162 5th Avenue, Manhattan, NY, USA}
\affiliation{Department of Astronomy, Columbia University, Manhattan, NY, USA}

\author{Yuxi (Lucy) Lu}
\affiliation{Department of Astronomy, Columbia University, Manhattan, NY, USA}
\affiliation{Department of Astrophysics, American Museum of Natural History,
200 Central Park West, Manhattan, NY, USA}

\author{Dan Foreman-Mackey}
\affiliation{Center for Computational Astrophysics, Flatiron Institute,
162 5th Avenue, Manhattan, NY, USA}

\author{Adrian M. Price-Whelan}
\affiliation{Center for Computational Astrophysics, Flatiron Institute,
162 5th Avenue, Manhattan, NY, USA}

\author{Jason Curtis}
\affiliation{Department of Astrophysics, American Museum of Natural History,
200 Central Park West, Manhattan, NY, USA}

\author{Emily Cunningham}
\affiliation{Center for Computational Astrophysics, Flatiron Institute,
162 5th Avenue, Manhattan, NY, USA}

\begin{abstract}
Gyrochronology, the method of inferring the age of a star from its rotation
period, could provide ages for billions of stars over the coming decade of
time-domain astronomy.
However, the gyrochronology relations remain poorly calibrated due to a lack
of precise ages for old, cool main-sequence stars.
Now however, with proper motion measurements from Gaia, Galactic kinematics
can be used as an age proxy, and the magnetic and rotational evolution of
stars can be examined in detail.
We demonstrate that kinematic ages, inferred from the velocity dispersions of
groups of stars, beautifully illustrate the time and mass-dependence of the
gyrochronology relations.
We use the kinematic ages of field stars, plus benchmark clusters and
    asteroseismic stars, to calibrate a new empirical Gaussian process
    gyrochronology relation, that fully captures the complex rotational
    evolution of cool dwarfs over a range of masses and ages.
We use cross validation to demonstrate that this relation accurately predicts
ages for GKM dwarfs.
\end{abstract}

\keywords{
Stellar Rotation ---
Stellar Evolution ---
Stellar Activity ---
Stellar Magnetic Fields ---
Low Mass Stars ---
Solar Analogs ---
Milky Way Dynamics
}

\section{Introduction}

% Motivation
Low mass dwarfs are the most common stars in the Milky Way, and their ages
could reveal the evolution of Galactic stellar populations and planetary
systems.
However, the ages of GKM stars are difficult to measure because their
luminosities and temperatures evolve slowly on the main sequence.
Fortunately, rotation-dating, or `gyrochronology’ provides a promising means
to measure precise ages for these cool dwarfs.
The rotation periods of these stars evolve relatively rapidly, and a fully
calibrated gyrochronology model that captures the time and mass-dependence of
stellar spin down could provide ages that are precise to within 20\% for
millions of Milky Way stars in the time-domain era \citep{epstein2014,
najita2016, angus2019}.

% What do we still not understand?
% Gyro is still poorly calibrated
With the thousands of new photometric rotation period measurements provided by
specialized ground and space-based missions \citep[particularly \kepler/\ktwo\
and \tess][]{borucki2010, howell2014, ricker2015}, we are making progress
towards the ultimate goal for rotation-dating: a fully calibrated
gyrochronology relation, applicable to GKM main-sequence stars of all ages.
A lack of low-mass and old calibration stars has previously limited the mass
and age coverage of gyrochronology relations, which can only be calibrated
using stars with precise age and rotation period measurements.
Historically, the calibration sample has been limited to open clusters and
asteroseismic stars.
 % because cluster members and Solar-like oscillators can be
% precisely dated via main-sequence turn off and asteroseismology.
Open clusters provide good mass coverage for young stars: rotation periods
have been measured for F through mid M dwarfs for stars in clusters with
precisely measured ages up to around 700 Myr.
Asteroseismic stars provide reasonable age coverage for hot stars: seismic
masses and photometric surface rotation periods have been measured for F, G
and early K dwarfs for stars as old as 10 Gyr.
However, neither asteroseismology nor cluster analysis can provide rotation
periods and ages for old, late K and M dwarfs.
In addition, cluster and asteroseismic stars generally provide sparse coverage
of the rotation period-effective temperature plane, and cannot reveal the
detailed evolution of stellar rotation rates.
As a result, most empirical gyrochronology relations are only reliable for G
dwarfs up to Solar age, K dwarfs up to 2-3 Gyr, and early M dwarfs up to $<$ 1
Gyr.
% Old open clusters are generally too distant for photometric rotation period
% measurements of faint M dwarfs, and asteroseismic analyses suffer from the
% large magnetic activity signals and low-amplitudes of oscillations for
% low-mass stars.
For this reason, the rotational evolution of cool dwarfs is not well
understood.
However, as we showed in \citet{angus2020}, {\it kinematic} ages can be used
to turn field stars, observed by \kepler, into a calibration sample that
provides good mass and age coverage.
Although the \kepler\ sample does not include late M dwarfs, it can still be
used to extend gyrochronology relations to much older ages for late K and
early M dwarfs.
By adding the ages and rotation periods of thousands of field stars to the
open cluster and asteroseismic calibration sample, we can calibrate a
gyrochronology relation that is applicable to FGK and early M dwarfs between
the ages of $\sim$500 Myr and 8 Gyr.

\begin{itemize}
\item{How are rotation periods measured?}
\end{itemize}

% % Gyro 101
% Stars with significant convective envelopes ($\lesssim$ 1.3 M$_\odot$) have
% strong magnetic fields which are thought to be generated by a Solar-type
% $\alpha-\Omega$ dynamo.
% These stars lose angular momentum over long timescales via magnetic braking
% \citep[\eg][]{schatzman1962, weber1967, kraft1967, skumanich1972, kawaler1988,
% pinsonneault1989}.
% Stars arrive on the main sequence with a random distribution of rotation
% periods, ranging from 1 oto 10 days \citep{rebull2019}, however the rotation
% periods of stars in open clusters have converged onto a unique age-rotation
% period-mass sequence by $\sim$500-700 million years \citep[\eg][]{irwin2009,
% gallet2013}.
% After this time, the rotation period of a star is thought to be determined, to
% first order, by its photometric color/effective temperature/mass and age
% \citep[\eg]{barnes2003, barnes2007, barnes2010, meibom2011, meibom2015}.

% The rate at which a star loses angular momentum is thought to chiefly depend
% on its magnetic field strength.
% Lower-mass stars with deeper convection zones have stronger magnetic fields,
% thus larger Alfvén radii, and therefore experience greater angular momentum
% loss rates \citep[\eg][]{schatzman1962, kraft1967, parker1970, kawaler1988,
% charbonneau2010}.
% This effect dominates the evolutionary sequence of young stars and, for
% example in the $\sim$ 700 Myr Praesepe cluster, rotation rate increases with
% increasing mass (rotation {\it period decreases} with increasing mass).
% However, at ages greater than around 1 Gyr \citep{spada2019, curtis2019,
% angus2020}, internal angular momentum transport becomes important and starts
% to influence the evolution of surface rotation periods.

% % Core-envelope coupling
% A period of core-envelope decoupling is necessary to explain the observed
% rotation periods of stars in extremely young open clusters (1-10 Gyr)
% \citep[\eg][]{irwin2007, bouvier2008, denissenkov2010, spada2011, reiners2012,
% gallet2013}.
% After the development of a radiative core, little angular momentum is
% transported between the core and convective envelope and, if wind-braking
% slows the surface substantially before the two zones recouple, the core may
% rotate much more quickly than the envelope.
% The Sun rotates almost as a solid body \citep[\eg][]{thompson1996}, so it is
% expected that the cores and envelopes of Solar-like stars eventually
% re-couple, with angular momentum efficiently transported between them.
% The timescales for recoupling have been studied extensively and explored in
% theoretical models for decades \citep[\eg][]{endal1981, macgregor1991,
% denissenkov2010, gallet2013, lanzafame2015}.
% The physical mechanism responsible for internal angular transport is still
% unknown, however magnetic field-induced coupling and gravity waves are two
% processes often used to explain the phenomenon \citep[see,
% \eg][]{charbonneau1993, ruediger1996, spruit2002, talon2003, spada2010,
% brun2011, oglethorpe2013}.
% Based on observations of lithium depletion and rotation period evolution in
% open clusters, including new rotation period measurements for the $\sim$1 Gyr
% NGC6811 cluster \citep{curtis2019}, the coupling timescale has been determined
% to be strongly mass-dependent \citep{lanzafame2015, somers2016, spada2019}.
% Semi-empirical models with mass-dependent angular momentum transport are able
% to reproduce the rotation periods in open clusters and the field
% \citep{spada2019, angus2020}.

% Summary of using velocity dispersions as an age proxy -- lit review
\subsection{Using kinematics as an age proxy}

% Describe previous paper
In general, the velocity dispersions of stars in the Galactic thick disk are
observed to increase with age \citep[\eg][]{casagrande2011, aumer2009}

In Angus \etal\ (2020) we demonstrated that Galactic kinematics can be used to
explore the evolution of stellar rotation.
We calculated the velocity dispersions of groups of stars with similar
rotation periods and effective temperatures to show that K dwarfs spin down
more slowly than G dwarfs.
Their rotational evolution appears to `stall' after around 1 Gyr, in a manner
that reflects the behavior of K dwarfs observed in open clusters
\citep{curtis2019}.
At young ages ($\sim$ 0.5 -- 1 Gyr), K dwarfs spin more slowly than G dwarfs of
the same age, because their deeper convection zones generate stronger magnetic
fields, which leads to more efficient magnetic braking.
However, at old ages ($\gtrsim$ 1 Gyr) K dwarfs rotate at the same rate, or more
rapidly than G dwarfs of the same age.
The leading theory for this phenomenon is that the angular momentum of the
radiative cores and convective envelopes of K dwarfs evolve separately at
young ages.
The envelope loses angular momentum via magnetic braking while the core
continues to spin rapidly.
Over time however, angular momentum is transported across the interface
between the two zones, and momentum from the rapidly spinning interior
surfaces, inhibiting the deceleration of the outer envelope.
With a mass-dependent timescale for core-envelope coupling, semi-empirical
models are able to reproduce the rotation periods of field and cluster stars
\citep[][Angus \etal, 2020]{spada2019, curtis2019}.

Most stars with rotation periods measured from \kepler\ data do not have RV
measurements\footnote{although RVs for most will be released in \gaia\ DR3},
so in Angus \etal\ (2020), we used \vb, velocity in the direction of Galactic
latitude, $b$, as a stand-in for \vz.
In the Galactic coordinate system, velocities can be calculated from 3D
positions: RA ($\alpha$), dec ($\delta$) and parallax ($\pi$) and 2D proper
motions, proper motion in RA (\mura) and dec (\mudec).
So \vb\ can be calculated without RV measurements, but is still a close
approximation to \vz\ for \kepler\ stars by virtue of its low Galactic
latitude.
Unfortunately however, given that AVRs are calibrated in {\it galactocentric}
coordinates (\vx, \vy, \vz), we could not directly translate \vb\ velocity
dispersions to ages.

In this paper, our aim was to use kinematic ages to calibrate a new
gyrochronology relation.
Three main technological improvements to the analysis applied in Angus \etal\
(2020) were required to calibrate this new relation.
Firstly, we calculated {\it vertical} velocity, \vz, rather than velocity in
the Galactic latitude direction, \vb, for each star.
We did this by inferring \vx, \vy, and \vz, while marginalizing over missing
RV measurements, using a hierarchical Bayesian model (see section
\ref{sec:velocity_inference}).
Secondly, instead of making a single estimate of the velocity dispersion of a
group of stars, we calculated the velocity dispersion of groups of stars,
centered on each star, \ie\ we calculated a moving, or rolling dispersion (see
section \ref{sec:velocity_dispersion}).
These velocity dispersions were then converted into ages using an AVR
\citep{yu2019} in section \ref{sec:avr}.
Thirdly, we used a Gaussian process model to capture the complexities of
stellar rotational evolution and calibrated a new GP-based gyrochronology
relation using our new kinematic ages, plus benchmark cluster and
asteroseismic stars in section \ref{sec:gp_model}.


\section{Method}

\begin{equation}
% p(v_{\bf x}, v_{\bf y}, v_{\bf z}, D | \mu_{\alpha}, \mu_{\delta}, \alpha,
% \delta, \pi) = p(\mu_{\alpha}, \mu_{\delta}, \alpha, \delta, \pi | v_{\bf x},
% v_{\bf y}, v_{\bf z}, D) p(v_{\bf x}) p(v_{\bf y}) p(v_{\bf z}) p(D),
p(v_{\bf xyz}, D | \mu_{\alpha}, \mu_{\delta}, \alpha, \delta, \pi) =
    p(\mu_{\alpha}, \mu_{\delta}, \alpha, \delta, \pi | v_{\bf xyz}, D)
    p(v_{\bf xyz}) p(D),
\end{equation}
where D is distance, $\alpha$ is Right Ascension (RA), $\delta$ is declination
(dec), $\pi$ is parallax, $\mu_\alpha$ is proper motion in RA, and
$\mu_\delta$ is proper motion in dec.



\section{Results}

\subsection{A Gaussian process gyrochronology relation}
\label{sec:gp_model}

To calibrate a new gyrochronology relation, we fit a Gaussian process-based
model to the kinematic ages described above, as well as a set of field and
cluster benchmark stars.
These benchmark stars are stars with precise rotation periods measured from
Kepler/K2 light curves, and well-determined ages from cluster isochrone
fitting or asteroseismology.



\include{discussion}

\include{conclusion}

\include{acknowledgements}

\bibliography{aviary}{}
\bibliographystyle{aasjournal}

\end{document}
