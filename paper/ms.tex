% zip files.zip ms63.tex CMD_cuts_double.pdf main_figure.pdf gap.pdf vb_vs_teff.pdf vb_vz.pdf field_comparison.pdf bib.bib ms.pdf

\documentclass{aastex63}

% Latin
\newcommand{\ie}{{\it i.e.}}
\newcommand{\eg}{{\it e.g.}}
\newcommand{\etal}{{\it et al.}}

% Missions & Spacecraft
\newcommand{\kepler}{{Kepler}}
\newcommand{\Kepler}{{Kepler}}
\newcommand{\corot}{{\it CoRoT}}
\newcommand{\Ktwo}{{\it K2}}
\newcommand{\ktwo}{\Ktwo}
\newcommand{\TESS}{{\it TESS}}
\newcommand{\tess}{{\it TESS}}
\newcommand{\LSST}{{\it LSST}}
\newcommand{\lsst}{{\it LSST}}
\newcommand{\Wfirst}{{\it WFIRST}}
\newcommand{\wfirst}{{\it WFIRST}}
\newcommand{\SDSS}{{\it SDSS}}
\newcommand{\PLATO}{{\it PLATO}}
\newcommand{\plato}{{\it PLATO}}
\newcommand{\Gaia}{{\it Gaia}}
\newcommand{\gaia}{{\it Gaia}}
\newcommand{\panstarrs}{{\it PanSTARRS}}
\newcommand{\LAMOST}{{\it LAMOST}}
\newcommand{\lamost}{{\it LAMOST}}

% Units and Quantities
\newcommand{\Teff}{$T_{\mathrm{eff}}$}
\newcommand{\teff}{$T_{\mathrm{eff}}$}
\newcommand{\FeH}{[Fe/H]}
\newcommand{\feh}{[Fe/H]}
\newcommand{\prot}{$P_{\mathrm{rot}}$}
\newcommand{\pmega}{$\bar{\omega}$}
\newcommand{\logg}{log(g)}
\newcommand{\dnu}{$\Delta \nu$}
\newcommand{\numax}{$\nu_{\mathrm{max}}$}
\newcommand{\degrees}{$^\circ$}
\newcommand{\vx}{$v_{\bf x}$}
\newcommand{\vy}{$v_{\bf y}$}
\newcommand{\vz}{$v_{\bf z}$}
\newcommand{\vb}{$v_{\bf b}$}
\newcommand{\x}{${\bf x}$}
\newcommand{\y}{${\bf y}$}
\newcommand{\z}{${\bf z}$}
\newcommand{\kms}{kms$^{-1}$}
\newcommand{\sigmavb}{$\sigma_{v{\bf b}}$}
\newcommand{\sigmavz}{$\sigma_{v{\bf z}}$}
\newcommand{\mura}{$\mu_\alpha$}
\newcommand{\pmra}{$\mu_\alpha$}
\newcommand{\mudec}{$\mu_\delta$}
\newcommand{\pmdec}{$\mu_\delta$}
\newcommand{\parallax}{$\pi$}
\newcommand{\ra}{$\alpha$}
\newcommand{\dec}{$\delta$}

% Paper-specific and other
\newcommand{\logp}{$\log(P_\mathrm{rot})$}
\newcommand{\gcolor}{$G_{BP} - G_{RP}$}
\newcommand{\mcp}{\citep{mcquillan2014}}
\newcommand{\mct}{\citet{mcquillan2014}}
\newcommand{\sant}{\citet{santos2019}}
\newcommand{\bvector}{${\bf b}$}
\newcommand{\python}{{\it Python}}
\newcommand{\racomment}[1]{{\color{blue}#1}}

% \shorttitle{Calibrating gyrochronology with kinematics}
% \shortauthors{Angus \etal}

\begin{document}

\title{Calibrating gyrochronology using Galactic kinematics}

% \correspondingauthor{Ruth Angus}
% \email{rangus@amnh.org}

\author{Ruth Angus}
\affiliation{Department of Astrophysics, American Museum of Natural History,
200 Central Park West, Manhattan, NY, USA}
\affiliation{Center for Computational Astrophysics, Flatiron Institute,
162 5th Avenue, Manhattan, NY, USA}
\affiliation{Department of Astronomy, Columbia University, Manhattan, NY, USA}

\author{Yuxi (Lucy) Lu}
\affiliation{Department of Astronomy, Columbia University, Manhattan, NY, USA}
\affiliation{Department of Astrophysics, American Museum of Natural History,
200 Central Park West, Manhattan, NY, USA}

\author{Dan Foreman-Mackey}
\affiliation{Center for Computational Astrophysics, Flatiron Institute,
162 5th Avenue, Manhattan, NY, USA}

\author{Adrian M. Price-Whelan}
\affiliation{Center for Computational Astrophysics, Flatiron Institute,
162 5th Avenue, Manhattan, NY, USA}

\author{Jason Curtis}
\affiliation{Department of Astrophysics, American Museum of Natural History,
200 Central Park West, Manhattan, NY, USA}

% \author{Emily Cunningham}
% \affiliation{Center for Computational Astrophysics, Flatiron Institute,
% 162 5th Avenue, Manhattan, NY, USA}

\begin{abstract}
Gyrochronology, the method of inferring the age of a star from its rotation
period, could provide ages for millions of stars in the Milky Way over the
    present and forthcoming era of time-domain astronomy.
Although significant progress has recently been made in calibrating empirical
    and semi-empirical gyrochronology models, they remain poorly calibrated
    for old, cool dwarfs due to a lack of appropriate calibration stars.
Now however, with proper motion measurements from Gaia, Galactic kinematics
can be used to calculate kinematic ages for ensembles of Milky Way disc stars.
% These kinematic ages can be calculated particularly an age proxy, and the
% magnetic and rotational evolution of stars can be examined in detail.
We use the kinematic ages of Kepler field stars with measured rotation
    periods, plus stars in open clusters, to calibrate a new fully empirical
    gyrochronology relation that captures the complex rotational evolution of
    cool dwarfs over a range of masses and ages.
We use a Gaussian process model to augment a power-law function in order to
    capture the time and color-dependence of rotational evolution.
We use cross validation to demonstrate that this relation predicts
    ages for the GKM dwarfs in our sample to within XX\%.
\end{abstract}

\keywords{
Stellar Rotation ---
Stellar Evolution ---
Stellar Activity ---
Stellar Magnetic Fields ---
Low Mass Stars ---
Solar Analogs ---
Milky Way Dynamics
}

\section{Introduction}

% Motivation
Low mass dwarfs are the most common stars in the Milky Way, and their ages
could reveal the evolution of Galactic stellar populations and planetary
systems.
However, the ages of GKM stars are difficult to measure because their
luminosities and temperatures evolve slowly on the main sequence.
Fortunately, rotation-dating, or `gyrochronology’ provides a promising means
to measure precise ages for these cool dwarfs.
The rotation periods of these stars evolve relatively rapidly, and a fully
calibrated gyrochronology model that captures the time and mass-dependence of
stellar spin down could provide ages that are precise to within 20\% for
millions of Milky Way stars in the time-domain era \citep{epstein2014,
najita2016, angus2019}.

% What do we still not understand?
% Gyro is still poorly calibrated
With the thousands of new photometric rotation period measurements provided by
specialized ground and space-based missions \citep[particularly \kepler/\ktwo\
and \tess][]{borucki2010, howell2014, ricker2015}, we are making progress
towards the ultimate goal for rotation-dating: a fully calibrated
gyrochronology relation, applicable to GKM main-sequence stars of all ages.
A lack of low-mass and old calibration stars has previously limited the mass
and age coverage of gyrochronology relations, which can only be calibrated
using stars with precise age and rotation period measurements.
Historically, the calibration sample has been limited to open clusters and
asteroseismic stars.
 % because cluster members and Solar-like oscillators can be
% precisely dated via main-sequence turn off and asteroseismology.
Open clusters provide good mass coverage for young stars: rotation periods
have been measured for F through mid M dwarfs for stars in clusters with
precisely measured ages up to around 700 Myr.
Asteroseismic stars provide reasonable age coverage for hot stars: seismic
masses and photometric surface rotation periods have been measured for F, G
and early K dwarfs for stars as old as 10 Gyr.
However, neither asteroseismology nor cluster analysis can provide rotation
periods and ages for old, late K and M dwarfs.
In addition, cluster and asteroseismic stars generally provide sparse coverage
of the rotation period-effective temperature plane, and cannot reveal the
detailed evolution of stellar rotation rates.
As a result, most empirical gyrochronology relations are only reliable for G
dwarfs up to Solar age, K dwarfs up to 2-3 Gyr, and early M dwarfs up to $<$ 1
Gyr.
% Old open clusters are generally too distant for photometric rotation period
% measurements of faint M dwarfs, and asteroseismic analyses suffer from the
% large magnetic activity signals and low-amplitudes of oscillations for
% low-mass stars.
For this reason, the rotational evolution of cool dwarfs is not well
understood.
However, as we showed in \citet{angus2020}, {\it kinematic} ages can be used
to turn field stars, observed by \kepler, into a calibration sample that
provides good mass and age coverage.
Although the \kepler\ sample does not include late M dwarfs, it can still be
used to extend gyrochronology relations to much older ages for late K and
early M dwarfs.
By adding the ages and rotation periods of thousands of field stars to the
open cluster and asteroseismic calibration sample, we can calibrate a
gyrochronology relation that is applicable to FGK and early M dwarfs between
the ages of $\sim$500 Myr and 8 Gyr.

\begin{itemize}
\item{How are rotation periods measured?}
\end{itemize}

% % Gyro 101
% Stars with significant convective envelopes ($\lesssim$ 1.3 M$_\odot$) have
% strong magnetic fields which are thought to be generated by a Solar-type
% $\alpha-\Omega$ dynamo.
% These stars lose angular momentum over long timescales via magnetic braking
% \citep[\eg][]{schatzman1962, weber1967, kraft1967, skumanich1972, kawaler1988,
% pinsonneault1989}.
% Stars arrive on the main sequence with a random distribution of rotation
% periods, ranging from 1 oto 10 days \citep{rebull2019}, however the rotation
% periods of stars in open clusters have converged onto a unique age-rotation
% period-mass sequence by $\sim$500-700 million years \citep[\eg][]{irwin2009,
% gallet2013}.
% After this time, the rotation period of a star is thought to be determined, to
% first order, by its photometric color/effective temperature/mass and age
% \citep[\eg]{barnes2003, barnes2007, barnes2010, meibom2011, meibom2015}.

% The rate at which a star loses angular momentum is thought to chiefly depend
% on its magnetic field strength.
% Lower-mass stars with deeper convection zones have stronger magnetic fields,
% thus larger Alfvén radii, and therefore experience greater angular momentum
% loss rates \citep[\eg][]{schatzman1962, kraft1967, parker1970, kawaler1988,
% charbonneau2010}.
% This effect dominates the evolutionary sequence of young stars and, for
% example in the $\sim$ 700 Myr Praesepe cluster, rotation rate increases with
% increasing mass (rotation {\it period decreases} with increasing mass).
% However, at ages greater than around 1 Gyr \citep{spada2019, curtis2019,
% angus2020}, internal angular momentum transport becomes important and starts
% to influence the evolution of surface rotation periods.

% % Core-envelope coupling
% A period of core-envelope decoupling is necessary to explain the observed
% rotation periods of stars in extremely young open clusters (1-10 Gyr)
% \citep[\eg][]{irwin2007, bouvier2008, denissenkov2010, spada2011, reiners2012,
% gallet2013}.
% After the development of a radiative core, little angular momentum is
% transported between the core and convective envelope and, if wind-braking
% slows the surface substantially before the two zones recouple, the core may
% rotate much more quickly than the envelope.
% The Sun rotates almost as a solid body \citep[\eg][]{thompson1996}, so it is
% expected that the cores and envelopes of Solar-like stars eventually
% re-couple, with angular momentum efficiently transported between them.
% The timescales for recoupling have been studied extensively and explored in
% theoretical models for decades \citep[\eg][]{endal1981, macgregor1991,
% denissenkov2010, gallet2013, lanzafame2015}.
% The physical mechanism responsible for internal angular transport is still
% unknown, however magnetic field-induced coupling and gravity waves are two
% processes often used to explain the phenomenon \citep[see,
% \eg][]{charbonneau1993, ruediger1996, spruit2002, talon2003, spada2010,
% brun2011, oglethorpe2013}.
% Based on observations of lithium depletion and rotation period evolution in
% open clusters, including new rotation period measurements for the $\sim$1 Gyr
% NGC6811 cluster \citep{curtis2019}, the coupling timescale has been determined
% to be strongly mass-dependent \citep{lanzafame2015, somers2016, spada2019}.
% Semi-empirical models with mass-dependent angular momentum transport are able
% to reproduce the rotation periods in open clusters and the field
% \citep{spada2019, angus2020}.

% Summary of using velocity dispersions as an age proxy -- lit review
\subsection{Using kinematics as an age proxy}

% Describe previous paper
In general, the velocity dispersions of stars in the Galactic thick disk are
observed to increase with age \citep[\eg][]{casagrande2011, aumer2009}

In Angus \etal\ (2020) we demonstrated that Galactic kinematics can be used to
explore the evolution of stellar rotation.
We calculated the velocity dispersions of groups of stars with similar
rotation periods and effective temperatures to show that K dwarfs spin down
more slowly than G dwarfs.
Their rotational evolution appears to `stall' after around 1 Gyr, in a manner
that reflects the behavior of K dwarfs observed in open clusters
\citep{curtis2019}.
At young ages ($\sim$ 0.5 -- 1 Gyr), K dwarfs spin more slowly than G dwarfs of
the same age, because their deeper convection zones generate stronger magnetic
fields, which leads to more efficient magnetic braking.
However, at old ages ($\gtrsim$ 1 Gyr) K dwarfs rotate at the same rate, or more
rapidly than G dwarfs of the same age.
The leading theory for this phenomenon is that the angular momentum of the
radiative cores and convective envelopes of K dwarfs evolve separately at
young ages.
The envelope loses angular momentum via magnetic braking while the core
continues to spin rapidly.
Over time however, angular momentum is transported across the interface
between the two zones, and momentum from the rapidly spinning interior
surfaces, inhibiting the deceleration of the outer envelope.
With a mass-dependent timescale for core-envelope coupling, semi-empirical
models are able to reproduce the rotation periods of field and cluster stars
\citep[][Angus \etal, 2020]{spada2019, curtis2019}.

Most stars with rotation periods measured from \kepler\ data do not have RV
measurements\footnote{although RVs for most will be released in \gaia\ DR3},
so in Angus \etal\ (2020), we used \vb, velocity in the direction of Galactic
latitude, $b$, as a stand-in for \vz.
In the Galactic coordinate system, velocities can be calculated from 3D
positions: RA ($\alpha$), dec ($\delta$) and parallax ($\pi$) and 2D proper
motions, proper motion in RA (\mura) and dec (\mudec).
So \vb\ can be calculated without RV measurements, but is still a close
approximation to \vz\ for \kepler\ stars by virtue of its low Galactic
latitude.
Unfortunately however, given that AVRs are calibrated in {\it galactocentric}
coordinates (\vx, \vy, \vz), we could not directly translate \vb\ velocity
dispersions to ages.

In this paper, our aim was to use kinematic ages to calibrate a new
gyrochronology relation.
Three main technological improvements to the analysis applied in Angus \etal\
(2020) were required to calibrate this new relation.
Firstly, we calculated {\it vertical} velocity, \vz, rather than velocity in
the Galactic latitude direction, \vb, for each star.
We did this by inferring \vx, \vy, and \vz, while marginalizing over missing
RV measurements, using a hierarchical Bayesian model (see section
\ref{sec:velocity_inference}).
Secondly, instead of making a single estimate of the velocity dispersion of a
group of stars, we calculated the velocity dispersion of groups of stars,
centered on each star, \ie\ we calculated a moving, or rolling dispersion (see
section \ref{sec:velocity_dispersion}).
These velocity dispersions were then converted into ages using an AVR
\citep{yu2019} in section \ref{sec:avr}.
Thirdly, we used a Gaussian process model to capture the complexities of
stellar rotational evolution and calibrated a new GP-based gyrochronology
relation using our new kinematic ages, plus benchmark cluster and
asteroseismic stars in section \ref{sec:gp_model}.

% % % Core-envelope coupling
% A period of core-envelope decoupling is necessary to explain the observed
% rotation periods of stars in extremely young open clusters (1-10 Gyr)
% \citep[\eg][]{irwin2007, bouvier2008, denissenkov2010, spada2011, reiners2012,
% gallet2013}.
% After the development of a radiative core, little angular momentum is
% transported between the core and convective envelope and, if wind-braking
% slows the surface substantially before the two zones recouple, the core may
% rotate much more quickly than the envelope.
% The Sun rotates almost as a solid body \citep[\eg][]{thompson1996}, so it is
% expected that the cores and envelopes of Solar-like stars eventually
% re-couple, with angular momentum efficiently transported between them.
% The timescales for recoupling have been studied extensively and explored in
% theoretical models for decades \citep[\eg][]{endal1981, macgregor1991,
% denissenkov2010, gallet2013, lanzafame2015}.
% The physical mechanism responsible for internal angular transport is still
% unknown, however magnetic field-induced coupling and gravity waves are two
% processes often used to explain the phenomenon \citep[see,
% \eg][]{charbonneau1993, ruediger1996, spruit2002, talon2003, spada2010,
% brun2011, oglethorpe2013}.
% Based on observations of lithium depletion and rotation period evolution in
% open clusters, including new rotation period measurements for the $\sim$1 Gyr
% NGC6811 cluster \citep{curtis2019}, the coupling timescale has been determined
% to be strongly mass-dependent \citep{lanzafame2015, somers2016, spada2019}.
% Semi-empirical models with mass-dependent angular momentum transport are able
% to reproduce the rotation periods in open clusters and the field
% \citep{spada2019, angus2020}.

\subsection{Core-envelope decoupling}

% Describe previous paper
In Angus \etal\ (2020) we demonstrated that Galactic kinematics can be used to
explore the evolution of stellar rotation.
We showed that velocity dispersion, an established age proxy in the Galactic
thin disk, increases smoothly as a function of rotation period, indicating
that rotation period increases with age as expected.
Using velocity dispersion as an age proxy, we also showed that old K dwarfs
spin down more slowly than G dwarfs: their rotational evolution appears to
`stall' after around 1 Gyr, in a manner that reflects the behavior of K dwarfs
observed in open clusters \citep{curtis2019}.
At young ages ($\sim$ 0.5 -- 1 Gyr), K dwarfs spin more slowly than G dwarfs of
the same age, because their deeper convection zones generate stronger magnetic
fields, which leads to more efficient magnetic braking.
However, at old ages ($\gtrsim$ 1 Gyr) K dwarfs rotate at the same rate or
more rapidly than contemporary G dwarfs.
The leading explanation for this phenomenon is that angular momentum is
transferred from the core to the surface over longer timescales for lower-mass
stars \citep{spada2019}, \ie\ they experience a more extended phase of
`core-envelope decoupling'.

% The angular momentum of the radiative cores and convective envelopes of stars
% are thought to evolve separately at young ages \citep[\eg][]{mcdonald1995,
% gallet2013}.
A period of core-envelope decoupling is necessary to explain the observed
rotation periods of stars in extremely young open clusters (1-10 Gyr)
\citep[\eg][]{irwin2007, bouvier2008, denissenkov2010, spada2011, reiners2012,
gallet2013}.
During this phase there is little transfer of angular momentum between
radiative core and convective envelope and, as wind-braking removes angular
momentum from the envelope, it decelerates while the core continues to spin
rapidly.
Over time however, angular momentum is transported across the interface
between the two zones, and momentum from the rapidly spinning interior
surfaces, inhibiting the deceleration of the outer envelope.
% With a mass-dependent timescale for core-envelope coupling semi-empirical
% models are able to reproduce the
Currently, the rotation periods of field and cluster stars can only be
reproduced by semi-empirical models with a mass-dependent timescale for
core-envelope coupling \citep[][Angus \etal, 2020]{spada2019, curtis2019}.


% % Summary of using velocity dispersions as an age proxy -- lit review
\subsection{Kinematic ages}

The star forming molecular gas clouds observed in the Milky Way have a low
out-of-plane, or vertical, velocity \citep[\eg][]{stark1989, stark2005,
aumer2009, martig2014, aumer2016}.
In contrast, the vertical velocities of older stars are observed to be larger
in magnitude on average \citep{stromberg1946, wielen1977, nordstrom2004,
holmberg2007, holmberg2009, aumer2009, casagrande2011, ting2019, yu2018}.
There are two possible explanations for this observed increase in velocity
dispersion with age: either stars are born kinematically `cool' and their
orbits are heated over time via interactions with giant molecular clouds
\citep[see][for a review of secular evolution in the MW]{sellwood2014}, or
stars formed kinematically `hotter' in the past \citep[\eg][]{bird2013}.
Either way, the vertical velocity dispersions of thin disk stars are observed
to increase with stellar age.
This behavior is codified by Age-Velocity dispersion Relations (AVRs), which
typically express the relationship between age and velocity dispersion as a
power law: $\sigma_v \propto t^\beta$, with free parameter, $\beta$
\citep[\eg][]{holmberg2009, yu2018}.
These expressions can be used to infer the ages of groups of stars from their
velocity dispersions, as we did in \citet{lu2021}

Kinematic ages have been used to explore the evolution of cool dwarfs for over
a decade.
\citet{west2004, west2006} found that the fraction of magnetically active M
dwarfs decreases over time, by using the vertical distances of stars from the
Galactic mid-plane as an age proxy, and \citet{west2008} used kinematic ages
to calculate the expected activity lifetime for M dwarfs of different spectral
types.
\citet{faherty2009} used tangential velocities to infer the ages of M, L and T
dwarfs, and showed that dwarfs with lower surface gravities tended to be
kinematically younger, and \citet{kiman2019} used velocity dispersion as an
age proxy to explore the evolution of H$\alpha$ equivalent width (a magnetic
activity indicator), in M dwarfs.

% After the development of a radiative core, little angular momentum is
% transported between the core and convective envelope and, if wind-braking
% slows the surface substantially before the two zones recouple, the core may
% rotate much more quickly than the envelope.
% The Sun rotates almost as a solid body \citep[\eg][]{thompson1996}, so it is
% expected that the cores and envelopes of Solar-like stars eventually
% re-couple, with angular momentum efficiently transported between them.
% The timescales for recoupling have been studied extensively and explored in
% theoretical models for decades \citep[\eg][]{endal1981, macgregor1991,
% denissenkov2010, gallet2013, lanzafame2015}.

AVRs are usually calibrated in Galactocentric velocity coordinates (\vx, \vy,
\vz\ or $UVW$), and these velocities can only be calculated with full 6D
positional and velocity information, however most \kepler\ rotators do not
have RV measurements\footnote{Although RVs for most will be released in \gaia\
DR3}.
In Angus \etal\ (2020) we used velocity in the direction of Galactic latitude
(\vb) as a stand-in for \vz\ because, in the {\it Galactic} coordinate system,
velocities can be calculated from 3D positions and {\it 2D} proper motions.
The \kepler\ field lies at low Galactic latitude, so \vb\ is a close
approximation to \vz.
Though \vb\ velocity dispersion does not equal \vz\ velocity dispersion, it
still increases monotonically over time and provides accurate age rankings for
\kepler\ stars.
Unfortunately however, given that AVRs are calibrated in {\it Galactocentric}
coordinates (\vx, \vy, \vz), we could not directly translate \vb\ velocity
dispersions to ages.

In this paper, our aim was to use kinematic ages to calibrate a new
gyrochronology relation, for which four main steps were required.
Firstly, we inferred {\it vertical} velocity, \vz, for each star without an RV
measurement by marginalizing over missing RVs using a hierarchical Bayesian
model (see section \ref{sec:velocity_inference}).
Secondly, we calculated velocity dispersion for every star using a moving, or
rolling dispersion method (see section \ref{sec:velocity_dispersion}).
Thirdly, these velocity dispersions were converted into ages using an AVR
\citep[][section \ref{sec:avr}]{yu2018}.
Finally, we used a Gaussian process model to capture the complexities of
stellar rotational evolution and calibrated a new gyrochronology relation
using our kinematic ages, plus benchmark cluster and asteroseismic stars in
section \ref{sec:gp_model}.


\section{The Data}
\label{sec:data}

This study focuses on stellar rotation in the original \kepler\ field, partly
because \kepler\ provides the largest samples of homogeneously measured
rotation periods, and partly because its low Galactic latitude allows us to
marginalize over missing RV measurements and precisely infer vertical
velocity, \vz.
We combined two large rotation period catalogs constructed from original
\kepler\ data: \mct\ and \sant.
These two studies used different techniques to measure rotation periods from
\kepler\ light curves: autocorrelation functions and wavelets respectively.
The \citet{santos2019} study was specifically focused on cooler stars: K and M
dwarfs, and includes a larger number of rotation periods for these stars.
The combined catalogs provided rotation periods for a total of 38,710 stars.

We used the publicly available \kepler-Gaia DR2 crossmatched
catalog\footnote{Available at gaia-kepler.fun} to combine the \mct\ and \sant\
rotation catalogs with the Gaia DR2 catalog of parallaxes, proper motions
and apparent magnitudes.
Reddening and extinction from dust was calculated for each star using the
Bayestar dust map implemented in the {\tt dustmaps} {\it Python} package
\citep{green2018}, and {\tt astropy} \citep{astropy2013, astropy2018}.
We used Gaia DR2 photometric color, $G_{\rm BP} - G_{\rm RP}$, to estimate
effective temperatures for the stars in our sample, using the calibrated
relation in \citet{curtis2020}.

% Unlike isolated main-sequence stars, the rotation periods of binary stars and
% subgiants cannot always be determined by their mass and age (or at least they
% do not always follow the {\it same} gyrochronology relationship as isolated
% dwarfs).
% Photometric binaries and subgiants were therefore removed from the sample by
% applying cuts to the color-magnitude diagram (CMD), shown in figure
% \ref{fig:CMD}.
% A 6th-order polynomial was fit to the main sequence and raised by 0.27 dex to
% approximate the division between single stars and photometric binaries (shown
% as the curved dashed line in figure \ref{fig:CMD}).
% All stars above this line were removed from the sample.
% Potential subgiants were also removed by eliminating stars brighter than 4th
% absolute magnitude in Gaia G-band.
% This cut also removed a number of main sequence F stars from our sample,
% however these hot stars are not the focus of our gyrochronology study since
% their small convective zones inhibit the generation of a strong magnetic
% field.
% The removal of photometric binaries and evolved/hot stars reduced the total
% sample of around 38,000 stars by around 4,000.

3705 stars in our sample had RV measurements available in Gaia DR2, with a
median uncertainty of 1.88 \kms.
Gaia DR2 included RVs for stars with Gaia apparent magnitudes between 4
and 13, and 3550 K $\lesssim$ \teff\ $\lesssim$ 6900 K \citep{brown2018}.
We also crossmatched the \mct\ sample with the 5th LAMOST data release
\citep{cui2012, xiang2019}, adding a further 10623 RV measurements to the
sample, and expanding the total number of stars with measured RVs to 14,328.
The median uncertainty of the LAMOST RV measurements was 4.71 \kms\ and,
given that the Gaia RVs were more precise, on average, than the LAMOST
RVs, we adopted the Gaia value in cases where both were available.
We note that the third Gaia data release will contain a large number of new
RV measurements for the stars in our sample.

We removed stars with a Gaia parallax signal-to-noise ratio of less than 10,
stars with negative parallaxes, and stars with a Gaia astrometric excess noise
value greater than 5.
After these cuts, 35,328 stars remained in our sample, of which 11,050 had RV
measurements from either Gaia or LAMOST.
We calculated 3D velocities for all stars with measured RVs using {\tt
astropy}, and {\it inferred} 3D velocities for the remaining 24,278 stars
using the method described in section \ref{sec:velocities}.


\section{Stellar Velocities}
\label{sec:velocities}

To calculate kinematic ages, an estimate of vertical velocity, \vz, is
required.
The ideal way to calculated \vz\, and similarly, \vx\ and \vy, is to use 6D
positional and velocity information.
Many stars in the \kepler\ field do not have RV measurements and an
alternative approach must be taken to infer their vertical velocities (see
section \ref{sec:velocity_inference}).
However, a large number of \kepler\ rotators, over 10,000 of 34,000 {\it do}
have RV measurements from \gaia\ DR2 and \lamost.
Figure \ref{fig:existing_rvs} shows rotation period vs effective temperature
for all stars in the \mct\ and \citet{santos2019} catalogs, plotted in grey.
Stars with RV measurements are colored by their vertical velocity dispersion
(see section \ref{sec:velocity_dispersion} to see how we calculated velocity
dispersion).
\racomment{Discuss what this plot shows}.

Although RVs are available for a significant number of \kepler\ rotators
(almost one in three), few stars cooler than 4000 K have RV measurements.
This is due to the faint limits of the \gaia\ DR2 and \lamost\ surveys
(although RV measurements for fainter targets will be available in \gaia\
DR3).
Given that magneto-rotational evolution is poorly understood for M dwarfs, the
cool stars with missing RVs are arguably the ones we care most about.
For this reason, we attempted to compensate for the lack of RV measurements by
inferring vertical velocities for stars without RVs, to fill in the
low-temperature region of figure \ref{fig:existing_rvs} in section
\ref{sec:velocity_inference}.
\begin{figure}[ht!]
\caption{
Vertical velocity dispersion as a function of rotation period and effective
    temperature for \kepler\ stars with measured rotation periods.
Colored points show stars with RV measurements from \gaia\ or \lamost, with
    their color indicating their velocity dispersion.
Faint grey points show the combined \mct\ and \citet{santos2019} samples,
    including stars without RV measurements.
The coolest stars in this sample do not have RVs because they are faint.
}
  \centering \includegraphics[width=1\textwidth]{existing_rvs}
\label{fig:existing_rvs}
\end{figure}


\section{Inferring 3D velocities (marginalizing over missing RV
measurements)}
\label{sec:inference}

It has been demonstrated that the dispersion in vertical velocity, \vz\, for a
group of stars increases with the age of that group (citations).
However, velocities in Galactocentric coordinates, \vx, \vy\ and \vz, can only
be calculated with full 6-D position and velocity information, \ie\ proper
motions, position and radial velocity.
In Angus \etal\ (2020) we showed that kinematic ages can be used to explore
rotational evolution and showed, in the appendix of that paper, that velocity,
\vb\ in the Galactic frame, which can be calculated without an RV measurement,
can be used as an approximation to \vz\ for \kepler\ stars.
This is because the \kepler\ field of view lies at relatively low Galactic
latitudes, ($\sim 5-20$\degrees), so the $z$-direction is similar to the
$b$-direction for \kepler\ stars.
However, \vb\ is only a close approximation to \vz\ at extremely {\it low}
latitudes, and even in the \kepler\ field, kinematic ages calculated with \vb\
instead of \vz\ are systematically larger because of extra noise introduced by
the imperfect translation between \vb\ and \vz .
In this work, we {\it infer} \vz\ by marginalizing over missing RV
measurements.

% Three-dimensional velocities in galactocentric coordinates: \vx, \vy, and \vz\
% can only be directly computed via a transformation from 3D velocities in
% another coordinate system, like the equatorial coordinates provided by \gaia:
% \mura, \mudec, and RV.
% For stars with no measured RV in \gaia\ DR2, \vx, vy, and \vz\ can still be
% inferred from positions and proper motions alone, by marginalizing over
% missing RV measurements.
For each star in our sample, we inferred \vx, \vy, and \vz\ from the 3D
positions and 2D proper motions provided in the \gaia\ DR2 catalog
\citep{brown2011}.
We also simultaneously inferred distance, (instead of using inverse-parallax),
to model velocities \citep[see \eg][]{bailer-jones2015, bailer-jones2018}.

Using Bayes rule, the posterior probability of the parameters given the data
can be written:
\begin{equation}
p(v_{\bf xyz}, D | \mu_{\alpha}, \mu_{\delta}, \alpha, \delta, \pi) =
    p(\mu_{\alpha}, \mu_{\delta}, \alpha, \delta, \pi | v_{\bf xyz}, D)
    p(v_{\bf xyz}) p(D),
\end{equation}
where D is distance, $\alpha$ is Right Ascension (RA), $\delta$ is declination
(dec), $\pi$ is parallax, $\mu_\alpha$ is proper motion in RA, and
$\mu_\delta$ is proper motion in dec.

For each star in the \kepler\ field, we explored the posteriors of these four
parameters using the {\it PyMC3} Hamiltonion Monte Carlo (HMC) sampler
\racomment{(citations)}.

\subsection{The prior}
\label{sec:prior}

\begin{itemize}
    \item{{\bf Why is the prior important?}
Arguably, the trickiest part of this inference is in selecting an appropriate
        prior.
Stellar velocities (in certain directions), inferred without RV measurements
        may be sensitive to the prior.
        }
    \item{{\bf How we calculated the prior.}
We used the mean and covariance of the distance and velocity distributions of
\kepler\ targets {\it with} RV measurements to determine the 4D multivariate
Gaussian prior over $\log$(distance) and velocities.
3D velocities were calculated for every star with an RV measurement from
either \gaia\ or \lamost.
These velocities were then sigma-clipped at the 3-sigma level in all three
dimensions to remove large velocity outliers which may be caused by proper
motion or RV measurements with large errors.
We then calculated the mean and covariance of the multivariate Gaussian
distribution of \vx, \vy, \vz, and $\ln$(Distance).
We used this mean and covariance to construct our multivariate Gaussian prior.
        }
    \item{{\bf Is is important to select the prior carefully.}
Our goal was to infer the velocities of stars without RV measurements using a
prior calculated from stars {\it with} RV measurements.
However, stars with and without RVs are likely to be slightly different
populations, the parameters of which depend on the \gaia\ and \lamost\
selection function.
\racomment{Discuss selection functions.}
In particular, stars without RV measurements are more likely to be faint, and
therefore, less massive.
Lower-mass stars are, on average, older, and have larger velocity dispersions.
So a prior based on the velocity distributions of stars with RVs will not
necessarily reflect the velocities of those without.
        }
    \item{{\bf Testing the prior.}
For this reason, we tested how the choice of prior affected the velocities we
inferred.
We tested two priors: one calculated from the velocity distributions of the
brightest half of the RV sample (\gaia\ $G$-band apparent magnitude $<$ 13.9),
and one from the faintest half ($G > $ 13.9).
We selected the 500 faintest stars from the \gaia-\lamost\ RV sample to test
these two priors on.
We chose the faintest stars as these are the most likely to be similar to the
non-RV sample, and to enlarge the difference between the bright prior and the
test sample.
The results of this test are shown in figure \ref{fig:prior_test}
        }
    \item{{\bf Although the prior isn't perfect, it's good
        enough for our study.}}
\end{itemize}


\begin{figure}[ht!]
\caption{
    }
  \centering
    \includegraphics[width=1\textwidth]{prior_distributions}
\label{fig:prior_distributions}
\end{figure}

We tuned the {\it PyMC3} model for 1500 steps, with a target acceptance
fraction of 0.9.
The model was then run for 1000 steps with 4 chains.

Figure \ref{fig:residuals} shows the \vx, \vy\ and \vz\ velocities we
inferred, compared with those calculated from measured RVs.
2000 stars are shown, which were chosen at random from the \kepler\ rotators
with RV measurements.
\begin{figure}[ht!]
\caption{Vertical velocities calculated with full 6D information vs vertical
    velocities inferred without RV, for all 3000 \mct\ stars with \gaia\ RV
    measurements.}
  \centering
    \includegraphics[width=1\textwidth]{residuals}
\label{fig:residuals}
\end{figure}
The \kepler\ field is oriented almost along the \y\-axis of the Galactocentric
coordinate system.
As a result, \x\ and \z-direction velocities of \kepler\ stars are extremely
well-constrained with proper motion alone, but \vy\ is almost completely
unconstrained without an RV.
Figure \ref{fig:v_comparison} shows that \vx\ and \vz\ velocities inferred
without RV measurements are extremely similar to those calculated with RVs.


% \section{Kinematic ages}
\subsection{Calculating velocity dispersions}
\label{sec:velocity_dispersion}

A kinematic age can be calculated from the velocity dispersion, \ie\ standard
deviation, of a group of stars.
These velocity dispersions can then be converted into an age using an AVR
\citep[\eg][]{holmberg2009, yu2018}.
Kinematic ages represent the {\it average age} of a group of stars and are
most informative when stars are grouped by age.
If a group of stars have similar ages, their kinematic age will be close
the age of each individual.
On the other hand, the kinematic age of a group with large age variance will
not provide much information about the ages of individual stars.
Velocity distributions themselves do not reveal whether a group of stars have
similar or different ages, since either case the velocities are
Gaussian-distributed.
Fortunately however, we can group \kepler\ stars by age using the implicit
assumption that underpins gyrochronology: that stars with the same rotation
period and color are the same age.
% We discuss the implications of this assumption and cases where it doesn't
% apply in the Discussion of this paper (section \ref{sec:discussion}).

In this paper, we use the kinematic ages published in \citet{lu2021}.
In that work, the kinematic age of each star in our sample was calculated by
placing it in a bin with other stars with similar rotation periods, effective
temperatures, absolute Gaia magnitudes and Rossby numbers.
The kinematic age of each star was estimated by calculating the velocity
dispersions of stars with these similar parameters, then using an AVR to
calculate a corresponding age \citep{yu2019}.
The bin size was optimized using a number of Kepler stars with asteroseismic
ages.

We used the \citet{yu2018} AVR to convert velocity dispersion to age.
This relation was calibrated using the ages and velocities of red clump stars.
They divided their sample into metal rich and poor subsets, and calibrated
separate AVRs for each, plus a global AVR.
Their AVR is a power law:
\begin{equation}
    \sigma_{vz} = \alpha t ^\beta,
\end{equation}
where $\alpha$ and $\beta$ take values (6.38, 0.578) for metal rich stars
(3.89, 1.01) for metal poor stars, and (5.47, 0.765) for all stars.

We used 1.5$\times$ the Median Absolute Deviation (MAD) of velocities, which
is a robust approximation to the standard deviation and is less sensitive to
outliers.
Velocity outliers could be binary stars or could be generated by
underestimated parallax or proper motion uncertainties.


\subsection{Calibrating a new gyrochronology relation}
\label{sec:calibration}

To calibrate a new gyrochronology relation, we fit a Gaussian process model to
the kinematic ages described above, as well as a number of stars in open
clusters.
These benchmark stars have precise rotation periods measured from Kepler/K2
light curves, and well-determined ages from cluster-based isochrone fitting.



\include{discussion}

\include{conclusion}

\include{acknowledgements}

\bibliography{aviary}{}
\bibliographystyle{aasjournal}

\end{document}
